\documentclass[11pt,a4paper]{google}

\usepackage[authoryear,sort&compress,round]{natbib}
\bibliographystyle{abbrvnat}

\usepackage[colorlinks=true,linkcolor=blue,urlcolor=blue,citecolor=blue]{hyperref}
\usepackage{booktabs}
\usepackage{longtable}
\usepackage{array}
\usepackage{verbatim}
\usepackage{xurl}
\usepackage{amsmath,amssymb,amsthm}
\usepackage{siunitx}
\usepackage{listings}
\usepackage{xcolor}
\usepackage{tikz}
\usetikzlibrary{arrows.meta,positioning,shapes.geometric}

\keywords{EEG, motor imagery, LaBraM, EEGNet, transfer learning, curriculum learning, BCI, reproducibility, negative results}

\title{EEG Motor Imagery with Foundation-Model Transfer and Compact CNN Baselines:\\
Comprehensive Consolidation of LaBraM Probing/Fine-Tuning/Curriculum and EEGNet Training/Tuning}

\author[1]{Prannaya Gupta}
\author[2]{Codex}
\author[3]{Rage Against The Machine Team}
\affil[1]{Independent Researcher}
\affil[2]{AI Research Assistant}
\affil[3]{RATM Project Team}

\correspondingauthor{prannaya.gupta@placeholder.edu}
\paperurl{https://github.com/your-org/your-repo}
\reportnumber{RATM-EEG-2026-02}

\newtheorem{proposition}{Proposition}
\newtheorem{corollary}{Corollary}
\newtheorem{remark}{Remark}

\lstdefinestyle{py}{
  language=Python,
  basicstyle=\ttfamily\footnotesize,
  keywordstyle=\color{blue!70!black},
  commentstyle=\color{green!35!black},
  stringstyle=\color{red!60!black},
  breaklines=true,
  showstringspaces=false,
  frame=single,
  tabsize=2
}

\newcommand{\ceThree}{\ensuremath{\log 3 \approx 1.0986}}
\newcommand{\ceTwo}{\ensuremath{\log 2 \approx 0.6931}}

\begin{abstract}
This manuscript consolidates the full EEG motor-imagery (MI) research trail implemented in \texttt{ml/}, including script code and notebook experiments (\texttt{ml/labram-for-eegmmidb.ipynb}). We systematize two modeling tracks: (i) LaBraM transfer experiments with probing, partial/full fine-tuning, and two-stage curriculum training; and (ii) EEGNet/EEGNetResidual compact baselines with subsequent user-level Muse calibration tuning. The cross-subject PhysioNet split used subjects 1--40 for training and 41--50 for validation, with runs [4, 8, 12]. Consolidated evidence shows persistent class-collapse dynamics for several LaBraM settings, especially in 3-class mode (left/right/rest), where completed runs plateaued at 0.5000 validation accuracy near \ceThree. Binary LaBraM improved modestly, with the best completed run reaching 0.5556. A full LaBraM fine-tune run reached train accuracy 0.9966 but only 0.5222 validation accuracy, indicating severe overfitting. The curriculum run improved checkpoint selection robustness (best MCC 0.0753; best val\_acc 0.5378; best balanced accuracy 0.5339) without surpassing the best binary accuracy. The compact EEGNetResidual baseline achieved 0.5919 on the cross-subject split. Muse personalization in current script defaults trains on all data (no holdout); when a holdout is enforced (\texttt{--no-use-all-data}), best validation accuracy is 0.5918 (epoch 82, 247 total epochs, 49 validation epochs). We provide formal propositions/proofs for diagnostic metrics (chance-level cross-entropy, collapse-accuracy bounds, MCC degeneracy for constant predictors, token-budget bounds in LaBraM preprocessing), exploratory code listings, and a reproducible run ledger suitable for critical academic audit.
\end{abstract}

\begin{document}
\maketitle

\section{Introduction}
EEG MI decoding remains difficult because discriminative information is weak relative to noise, highly nonstationary across sessions and subjects, and sensitive to preprocessing and data partitioning \citep{lawhern2018eegnet,schirrmeister2017deep}. Foundation-model (FM) transfer is attractive in principle: pre-train large representations once, then adapt to downstream tasks. In practice, reliable transfer requires that the downstream optimization interface preserve class-separable geometry under realistic channel and data constraints.

This work answers a practical question: \emph{what does the current project actually demonstrate when all \texttt{ml/} artifacts are jointly considered}? The previous report did not fully integrate notebook and script evidence. Here we consolidate these artifacts into one reproducible, technically defensible record.

\section{Consolidated Artifact Map}
Table~\ref{tab:artifact-map} maps code artifacts to scientific roles.

\begin{longtable}{@{}p{0.30\linewidth}p{0.65\linewidth}@{}}
\caption{Consolidated artifact map for \texttt{ml/} and notebook work.}\label{tab:artifact-map}\\
\toprule
Artifact & Role in this study \\
\midrule
\endfirsthead
\toprule
Artifact & Role in this study \\
\midrule
\endhead
\texttt{ml/eeg/dataset.py} & PhysioNet acquisition, event filtering, epoch creation, band-pass preprocessing. \\
\texttt{ml/models/labram\_probe.py} & Script-level LaBraM probe with optional block unfreezing and token pooling. \\
\texttt{ml/labram\_probe.py} & Reproducible LaBraM training entrypoint over subject-level split. \\
\texttt{ml/labram-for-eegmmidb.ipynb} & Main exploratory run trail: 3-class, binary, full fine-tune, curriculum, diagnostics, and embedding geometry analysis. \\
\texttt{ml/training\_notebook\_cell.py} & Scripted extraction of the notebook training logic: stage switch, MCC selection, per-class diagnostics, scheduler behavior. \\
\texttt{ml/eeg/layers/labram\_encoder.py} & Tokenization interface to pre-trained LaBraM backbone, including patch budget truncation rule. \\
\texttt{ml/eeg/layers/labram/neural\_transformer.py} & Backbone internals (patch size, depth, embedding path, class token). \\
\texttt{ml/\_\_main\_\_.py} & Cross-subject EEGNet/EEGNetResidual baseline training flow and summary logging. \\
\texttt{ml/tune\_eegnet\_muse.py} & User-level Muse fine-tuning script; includes all-data mode and optional holdout mode. \\
\texttt{docs/eeg/appendix\_labram\_raw\_logs.tex} & Verbatim log archive used for auditability and traceback preservation. \\
\bottomrule
\end{longtable}

\section{Materials and Methods}
\subsection{Datasets and Splits}
\textbf{Cross-subject PhysioNet MI setup.}
Runs [4, 8, 12] were used with events \texttt{T0}, \texttt{T1}, \texttt{T2} (rest, left, right), following the code configuration.
Observed consolidated split statistics:
\begin{itemize}
\item 3-class train: 3510 epochs, shape \((3510,64,601)\), counts \([884,871,1755]\).
\item 3-class val: 900 epochs, shape \((900,64,601)\), counts \([229,221,450]\).
\item Binary derivation (\texttt{T1}/\texttt{T2} only): train \([884,871]\), val \([229,221]\), totals 1755/450.
\end{itemize}

\textbf{Muse calibration setup.}
Current discovered calibration files:
\texttt{left\_muse\_v1,v2,v3.csv} and \texttt{right\_muse\_v1,v2,v3.csv} (6 files total).
Epoch counts (3-second windows) sum to 247 samples. Holdout mode with \texttt{val\_split=0.2} gives 198 training and 49 validation samples.

\subsection{Signal Processing}
Both tracks apply 8--30 Hz band-pass filtering (5th-order Butterworth in code).
Let \(x_{i,c}(t)\) be epoch \(i\), channel \(c\). Filtered signal is
\begin{equation}
\tilde{x}_{i,c}(t) = (h * x_{i,c})(t),
\end{equation}
with \(h\) induced by \((\text{lowcut},\text{highcut})=(8,30)\) Hz.

For LaBraM experiments, optional long-context interpolation resamples epochs to length \(L'=1600\):
\begin{equation}
\hat{x}_{i,c} = \mathcal{R}_{L'}(\tilde{x}_{i,c}),
\end{equation}
followed by per-epoch/channel z-score normalization.

For EEGNet tuning, Muse epochs are resampled to the checkpoint input length (\(n\_samples=481\)).

\subsection{Modeling Tracks}
\textbf{LaBraM transfer track.}
The notebook explored:
\begin{itemize}
\item frozen/probe-like training,
\item partial unfreezing of final transformer blocks,
\item full fine-tuning (\texttt{full\_finetune=True}),
\item curriculum: stage-1 head-only then stage-2 unfreeze-last-2 blocks, with MCC-based checkpoint selection.
\end{itemize}
The backbone in code uses \(T=\)200 patch size, \(d=\)200 embedding dimension, depth 12.

\textbf{EEGNet track.}
\begin{itemize}
\item Cross-subject baseline: \texttt{ml/\_\_main\_\_.py} with EEGNetResidual, 4 channels, best val accuracy 0.5919 (log excerpt in appendix).
\item Later tuning: \texttt{ml/tune\_eegnet\_muse.py} on user calibration CSVs.
\end{itemize}

\subsection{Optimization Objective}
Main classification objective:
\begin{equation}
\mathcal{L}_{\text{CE}} = -\frac{1}{B}\sum_{i=1}^{B}\log p_\theta(y_i|x_i).
\end{equation}
Some notebook runs used label smoothing (\(\epsilon=0.05\)); curriculum selection used MCC (binary) or balanced accuracy fallback.

\section{Formal Diagnostic Propositions and Proofs}
\subsection{Chance-level Cross-Entropy Baseline}
\begin{proposition}
For \(K\)-class classification with uniform prediction \(p(y=k|x)=1/K\), expected cross-entropy equals \(\log K\).
\end{proposition}
\begin{proof}
For any ground-truth class \(y\), loss is \(-\log p(y|x) = -\log(1/K)=\log K\). Expectation over data leaves \(\log K\) unchanged.
\end{proof}

\begin{corollary}
Random-guess baselines here are \ceThree\ for 3-class and \ceTwo\ for binary.
\end{corollary}

\subsection{Collapse Accuracy Bound}
\begin{proposition}
Let class priors be \(\pi_1,\dots,\pi_K\), \(\sum_j \pi_j=1\). If a classifier collapses to constant prediction \(\hat{y}=j^\star\), then accuracy is \(\pi_{j^\star}\). Best possible constant-prediction accuracy is \(\max_j \pi_j\).
\end{proposition}
\begin{proof}
Accuracy is \(\Pr(\hat{y}=y)=\Pr(y=j^\star)=\pi_{j^\star}\). Maximizing over \(j^\star\) yields \(\max_j \pi_j\).
\end{proof}

\begin{corollary}
For 3-class validation counts \([229,221,450]\), \(\max_j \pi_j = 450/900 = 0.5\). Thus persistent 0.5000 accuracy is compatible with majority-class collapse.
\end{corollary}

\subsection{MCC Degeneracy Under Constant Binary Prediction}
\begin{proposition}
Assume both binary classes appear in ground truth. If predictions are constant (\(\hat{y}\equiv 0\) or \(\hat{y}\equiv 1\)), then MCC is 0 under the standard safe-convention used in code (return 0 when denominator is 0).
\end{proposition}
\begin{proof}
MCC:
\[
\mathrm{MCC}=\frac{TP\cdot TN-FP\cdot FN}
{\sqrt{(TP+FP)(TP+FN)(TN+FP)(TN+FN)}}.
\]
For constant prediction, at least one confusion-matrix marginal in the denominator is zero, yielding denominator 0. The implemented function returns 0 in this case. Therefore MCC = 0.
\end{proof}

\begin{corollary}
In the binary validation split \([229,221]\), predicting all class 0 gives accuracy \(229/450=0.5089\), balanced accuracy 0.5, MCC 0.
\end{corollary}

\subsection{LaBraM Token Budget Bound}
\begin{proposition}
Given input sequence length \(SL\) and patch size \(T=200\), the implemented LaBraM encoder uses
\[
P = \min\!\left(\left\lfloor \frac{SL}{T}\right\rfloor,16\right)
\]
patches per channel. Patch-token count is \(N\cdot P\); with class token it is \(N\cdot P + 1\).
\end{proposition}
\begin{proof}
Directly from \texttt{ml/eeg/layers/labram\_encoder.py}: \(P\) is computed as above, the input is sliced to the last \(P\cdot T\) samples, then reshaped to \((B,N,P,T)\). The transformer appends one class token before attention.
\end{proof}

\begin{corollary}
Without long-context resampling, \(SL=601\Rightarrow P=3\). With resampling to 1600, \(P=8\). Hence long-context mode increases per-channel patch count from 3 to 8.
\end{corollary}

\section{Experimental Ledger}
\subsection{LaBraM Notebook Run Ledger}
Table~\ref{tab:labram-ledger} separates completed and interrupted runs.

\begin{longtable}{@{}p{0.07\linewidth}p{0.16\linewidth}p{0.17\linewidth}p{0.22\linewidth}p{0.14\linewidth}p{0.18\linewidth}@{}}
\caption{Consolidated LaBraM run ledger from \texttt{ml/labram-for-eegmmidb.ipynb}.}\label{tab:labram-ledger}\\
\toprule
Cell & Task & Core setup & Selection/diagnostics & Status & Best reported result \\
\midrule
\endfirsthead
\toprule
Cell & Task & Core setup & Selection/diagnostics & Status & Best reported result \\
\midrule
\endhead
10 & 3-class & Baseline-style training & Standard val acc & Completed & val\_acc 0.5000, early stop epoch 44 \\
12 & 3-class & Unfreeze-last-2, detailed per-class logs & Pred-count and per-class acc each epoch & Completed & val\_acc 0.5000, early stop epoch 41 \\
11 & 3-class & Partial unfreeze run & Standard diagnostics & Interrupted & best seen val\_acc 0.2543 (epoch 3) \\
13 & 3-class & CLS pooling, freeze config variant & Includes sanity probe & Interrupted & best seen val\_acc 0.4944 (sanity 0.5022) \\
7 & Binary & Full fine-tune; long-context 1600; label smoothing 0.05; CLS pooling & Frozen-embedding sanity probe & Completed & val\_acc 0.5222 (sanity 0.5444), early stop epoch 46 \\
8 & Binary & Curriculum: stage-1 head-only (12 epochs) then unfreeze last 2 blocks & MCC-based selection, per-class diagnostics & Completed & val\_acc 0.5378, bal\_acc 0.5339, MCC 0.0753, early stop epoch 53 (sanity 0.5600) \\
14 & Binary & Unfreeze-last-2, no long-context variant & Standard diagnostics & Interrupted & best seen val\_acc 0.5400 (epoch 74) \\
15 & Binary & Unfreeze-last-2 + long-context + label smoothing 0.05 & Standard diagnostics & Completed & val\_acc 0.5556, early stop epoch 83 \\
\bottomrule
\end{longtable}

\subsection{EEGNet Baseline and Tuning Ledger}
\begin{longtable}{@{}p{0.20\linewidth}p{0.33\linewidth}p{0.20\linewidth}p{0.22\linewidth}@{}}
\caption{EEGNet/EEGNetResidual results consolidated from script logs and direct reruns.}\label{tab:eegnet-ledger}\\
\toprule
Run & Data regime & Key setting & Outcome \\
\midrule
\endfirsthead
\toprule
Run & Data regime & Key setting & Outcome \\
\midrule
\endhead
Cross-subject EEGNetResidual baseline (\texttt{ml/\_\_main\_\_.py}) & Train 2925 / Val 750, shape \((4,481)\), class counts train \([1470,1455]\), val \([379,371]\) & Early-stop patience 200, CPU log excerpt archived & Best val\_acc 0.5919 at epoch 73; stop at epoch 273 \\
Muse tuning holdout (\texttt{tune\_eegnet\_muse.py --no-use-all-data}) & 6 CSVs, total 247 epochs, split 198/49 & Freeze early layers (\texttt{conv1}, \texttt{batchnorm1}), 200 epochs & Best val\_acc 0.5918 at epoch 82 \\
Muse tuning default (\texttt{tune\_eegnet\_muse.py}) & 6 CSVs, all 247 epochs in train & \texttt{--use-all-data} default enabled, no validation split & Best train loss 0.4539 at epoch 196; no validation accuracy reported \\
\bottomrule
\end{longtable}

\section{Results}
\subsection{LaBraM: Main Observations}
\textbf{3-class regime.}
Completed 3-class runs repeatedly converged to 0.5000 validation accuracy with strong collapse/flip behavior in prediction histograms (e.g., all predictions in one class, then another), consistent with Proposition 2 and corollary bounds.

\textbf{Binary regime.}
Best completed binary result was 0.5556 (cell 15), approximately +4.67 percentage points above majority-class collapse (0.5089), but still far below robust decoding thresholds typically expected for stable BCI control.

\textbf{Overfitting in full fine-tune (cell 7).}
Train accuracy increased to 0.9966 (epoch 42), while validation accuracy peaked at 0.5222 and validation loss rose above 1.5, indicating memorization without generalization.

\textbf{Curriculum behavior (cell 8).}
Stage switch at epoch 13 improved checkpoint selection quality using MCC (best 0.0753). This improved metric robustness but did not exceed the best binary accuracy obtained in cell 15.

\subsection{Embedding Geometry Probe}
From notebook exploratory analysis:
\begin{itemize}
\item PCA explained variance ratio: [0.19062445, 0.10972243].
\item Silhouette (PCA 2D): 0.0099735.
\item Silhouette (t-SNE 2D): 0.0099501.
\end{itemize}
Near-zero silhouette values indicate weakly separated clusters in low-dimensional projections, consistent with observed collapse dynamics.

\subsection{EEGNet Track Interpretation}
The compact cross-subject EEGNetResidual baseline (0.5919) exceeds all completed LaBraM runs in this consolidated record (best LaBraM 0.5556). Muse tuning with enforced holdout produced 0.5918. Importantly, current script defaults do not produce a validation metric unless \texttt{--no-use-all-data} is set, which is critical for scientific reporting discipline.

\section{Exploratory Code Listings}
\subsection*{Listing 1: LaBraM token-budget rule (\texttt{ml/eeg/layers/labram\_encoder.py})}
\begin{lstlisting}[style=py]
T = self.model.patch_size
P = min(SL // T, 16)
x = x[:, :, -P*T:]
x = rearrange(x, "B N (P T) -> B N P T", T=T)
\end{lstlisting}

\subsection*{Listing 2: Curriculum + MCC logic (\texttt{ml/training\_notebook\_cell.py})}
\begin{lstlisting}[style=py]
def binary_mcc_from_tensors(y_pred, y_true):
    tp = int(((y_pred == 1) & (y_true == 1)).sum().item())
    tn = int(((y_pred == 0) & (y_true == 0)).sum().item())
    fp = int(((y_pred == 1) & (y_true == 0)).sum().item())
    fn = int(((y_pred == 0) & (y_true == 1)).sum().item())
    denom = np.sqrt(float((tp + fp) * (tp + fn) * (tn + fp) * (tn + fn)))
    if denom == 0.0:
        return 0.0
    return float((tp * tn - fp * fn) / denom)

if stage2_enabled and epoch == stage2_start_epoch:
    model.configure_trainable_encoder(
        freeze_encoder=True,
        unfreeze_last_n_blocks=stage2_unfreeze_last_n_blocks,
    )
\end{lstlisting}

\subsection*{Listing 3: Muse split-mode guard (\texttt{ml/tune\_eegnet\_muse.py})}
\begin{lstlisting}[style=py]
if use_all_data:
    X_train, y_train = X, y
    X_val = np.empty((0, X.shape[1], X.shape[2]), dtype=X.dtype)
    y_val = np.empty((0,), dtype=y.dtype)
    print("Training mode: using all data (no validation split).")
else:
    X_train, y_train, X_val, y_val = _split_train_val(
        X, y, val_split=args.val_split, seed=args.seed
    )
    print(f"Training mode: train/val split with val_split={args.val_split:.2f}")
\end{lstlisting}

\section{Discussion}
The consolidated evidence supports a careful, publication-grade negative-to-mixed conclusion:
\begin{itemize}
\item In this project state, LaBraM transfer did not reliably surpass compact EEGNet baselines on cross-subject MI decoding.
\item Multiple LaBraM runs show collapse dynamics predicted by simple metric theory (Sections 4--6).
\item Curriculum learning improved checkpoint \emph{selection quality} (MCC), but not final best accuracy relative to the best binary run.
\item Muse personalization remains promising, but must be reported with explicit split protocol; all-data training cannot be treated as validation evidence.
\end{itemize}

These findings are still scientifically useful: they isolate where transfer currently fails (representation-task interface and optimization stability), rather than only reporting top-line numbers.

\section{Limitations and Threats to Validity}
\begin{itemize}
\item Several notebook runs were interrupted; they provide trajectory evidence, not completed endpoint claims.
\item No multi-seed inferential statistics are included yet.
\item Cross-subject and within-user regimes are different tasks; direct score comparison should be interpreted cautiously.
\item Muse holdout is random within captured sessions; external-session generalization remains untested.
\end{itemize}

\section{Conclusion}
After consolidating all available \texttt{ml/} artifacts, the current evidence shows: (i) frequent LaBraM class-collapse behavior in 3-class MI; (ii) modest binary improvements with best completed LaBraM val\_acc 0.5556; (iii) strong overfitting under full fine-tuning; (iv) improved selection stability but limited headline gain under curriculum learning; and (v) competitive compact CNN performance (0.5919 cross-subject baseline, 0.5918 Muse holdout tuning). The immediate research direction is not broader claim inflation, but rigorous stabilization: seed-robust protocols, stronger regularization, tighter calibration evaluation, and controlled comparisons against compact baselines under identical splits.

\section{Reproducibility and Availability}
\textbf{Code paths.}
Core scripts are in \texttt{ml/}; this manuscript references exact files listed in Table~\ref{tab:artifact-map}.

\textbf{Representative commands.}
\begin{verbatim}
python ml/labram_probe.py
python ml/tune_eegnet_muse.py --data-dir ml/calibration_data --no-use-all-data
python ml/tune_eegnet_muse.py --data-dir ml/calibration_data
\end{verbatim}

\textbf{Raw logs.}
Verbatim run excerpts and traceback archive are retained in Appendix~\ref{appendix:rawlogs}.

\section{Ethics and Safety}
This work is non-clinical and does not make medical claims. No personally identifying data are reported in this manuscript.

\bibliography{main}

\clearpage
\appendix
\section{Raw Experimental Logs and Error Trace}\label{appendix:rawlogs}
\section{Full Raw Logs (Verbatim Archive)}

\subsection{Run A: Initial 3-Class Log Block (Epoch 1--26)}
\scriptsize
\begin{verbatim}
Epoch 001/200 | Train loss=1.0869 acc=0.4662 | Val loss=1.0430 acc=0.5000
  New best (0.5000) saved to /kaggle/working/models/labram_probe/labram_probe_best.pth
Epoch 002/200 | Train loss=1.0492 acc=0.4986 | Val loss=1.0420 acc=0.5000
Epoch 003/200 | Train loss=1.0481 acc=0.4997 | Val loss=1.0423 acc=0.5000
Epoch 004/200 | Train loss=1.0447 acc=0.4996 | Val loss=1.0408 acc=0.5000
Epoch 005/200 | Train loss=1.0460 acc=0.5005 | Val loss=1.0440 acc=0.5000
Epoch 006/200 | Train loss=1.0443 acc=0.5001 | Val loss=1.0431 acc=0.5000
Epoch 007/200 | Train loss=1.0471 acc=0.4997 | Val loss=1.0403 acc=0.5000
Epoch 008/200 | Train loss=1.0427 acc=0.5003 | Val loss=1.0404 acc=0.5000
Epoch 009/200 | Train loss=1.0459 acc=0.5000 | Val loss=1.0410 acc=0.5000
Epoch 010/200 | Train loss=1.0457 acc=0.5000 | Val loss=1.0449 acc=0.5000
Epoch 011/200 | Train loss=1.0435 acc=0.5008 | Val loss=1.0401 acc=0.5000
Epoch 012/200 | Train loss=1.0425 acc=0.4996 | Val loss=1.0446 acc=0.5000
Epoch 013/200 | Train loss=1.0438 acc=0.5004 | Val loss=1.0415 acc=0.5000
Epoch 014/200 | Train loss=1.0469 acc=0.4997 | Val loss=1.0398 acc=0.5000
Epoch 015/200 | Train loss=1.0445 acc=0.4999 | Val loss=1.0410 acc=0.5000
Epoch 016/200 | Train loss=1.0433 acc=0.5000 | Val loss=1.0397 acc=0.5000
Epoch 017/200 | Train loss=1.0456 acc=0.5001 | Val loss=1.0402 acc=0.5000
Epoch 018/200 | Train loss=1.0447 acc=0.4999 | Val loss=1.0452 acc=0.5000
Epoch 019/200 | Train loss=1.0447 acc=0.4999 | Val loss=1.0398 acc=0.5000
Epoch 020/200 | Train loss=1.0435 acc=0.4997 | Val loss=1.0483 acc=0.5000
Epoch 021/200 | Train loss=1.0451 acc=0.5000 | Val loss=1.0463 acc=0.5000
Epoch 022/200 | Train loss=1.0422 acc=0.4999 | Val loss=1.0399 acc=0.5000
Epoch 023/200 | Train loss=1.0443 acc=0.5004 | Val loss=1.0436 acc=0.5000
Epoch 024/200 | Train loss=1.0425 acc=0.5001 | Val loss=1.0409 acc=0.5000
Epoch 025/200 | Train loss=1.0416 acc=0.5000 | Val loss=1.0407 acc=0.5000
Epoch 026/200 | Train loss=1.0415 acc=0.5005 | Val loss=1.0397 acc=0.5000
\end{verbatim}
\normalsize


\subsection{Run H: EEGNetResidual Baseline Log Excerpt (\texttt{ml/\_\_init\_\_.py})}
\scriptsize
\begin{verbatim}
============================================================
DATASET SUMMARY
============================================================
  Train: 2925 epochs (2925, 4, 481)  classes=[1470 1455]
  Val:   750 epochs (750, 4, 481)  classes=[379 371]
  Labels: 0=left hand, 1=right hand

============================================================
TRAINING MODEL
============================================================

  Device: cpu
  Model:  EEGNetResidual
  Epochs: 1000  |  Early-stop patience: 200

Epoch 001/1000 | Train loss=0.7035 acc=0.5148 | Val loss=0.6934 acc=0.4938
Epoch 007/1000 | Train loss=0.6906 acc=0.5270 | Val loss=0.6926 acc=0.5299
Epoch 015/1000 | Train loss=0.6832 acc=0.5566 | Val loss=0.6930 acc=0.5365
Epoch 023/1000 | Train loss=0.6810 acc=0.5634 | Val loss=0.6869 acc=0.5518
Epoch 032/1000 | Train loss=0.6757 acc=0.5691 | Val loss=0.6816 acc=0.5580
Epoch 045/1000 | Train loss=0.6665 acc=0.5926 | Val loss=0.6780 acc=0.5809
Epoch 059/1000 | Train loss=0.6613 acc=0.5938 | Val loss=0.6751 acc=0.5885
Epoch 073/1000 | Train loss=0.6564 acc=0.6095 | Val loss=0.6770 acc=0.5919
  -> New best (0.5919)
...
Epoch 273/1000 | Train loss=0.6105 acc=0.6523 | Val loss=0.7027 acc=0.5643

Early stopping at epoch 273.
Training complete. Best val acc: 0.5919
\end{verbatim}
\normalsize

\subsection{Run B/C: Partial Unfreeze (2 blocks and 4 blocks)}
\scriptsize
\begin{verbatim}
Epoch 001/200 | Train loss=1.1468 acc=0.3321 | Val loss=1.1015 acc=0.2457
  New best (0.2457) saved to /kaggle/working/models/labram_probe/labram_probe_best.pth
Epoch 002/200 | Train loss=1.1103 acc=0.3623 | Val loss=1.0987 acc=0.2457
Epoch 003/200 | Train loss=1.1001 acc=0.2557 | Val loss=1.0987 acc=0.2543
  New best (0.2543) saved to /kaggle/working/models/labram_probe/labram_probe_best.pth
Epoch 004/200 | Train loss=1.0998 acc=0.3086 | Val loss=1.0987 acc=0.2457
Epoch 005/200 | Train loss=1.0989 acc=0.2485 | Val loss=1.0987 acc=0.2457
Epoch 006/200 | Train loss=1.0992 acc=0.2826 | Val loss=1.0986 acc=0.2543
Epoch 007/200 | Train loss=1.0988 acc=0.2801 | Val loss=1.0986 acc=0.5000
  New best (0.5000) saved to /kaggle/working/models/labram_probe/labram_probe_best.pth
Epoch 008/200 | Train loss=1.0987 acc=0.3364 | Val loss=1.0986 acc=0.5000
Epoch 009/200 | Train loss=1.0993 acc=0.4317 | Val loss=1.0987 acc=0.2457
Epoch 010/200 | Train loss=1.0986 acc=0.4933 | Val loss=1.0987 acc=0.5000
Epoch 012/200 | Train loss=1.0989 acc=0.4741 | Val loss=1.0987 acc=0.5000
Epoch 013/200 | Train loss=1.0986 acc=0.5001 | Val loss=1.0987 acc=0.5000
Epoch 014/200 | Train loss=1.0986 acc=0.4986 | Val loss=1.0987 acc=0.5000
Epoch 015/200 | Train loss=1.0988 acc=0.4825 | Val loss=1.0987 acc=0.5000
Epoch 016/200 | Train loss=1.0987 acc=0.5005 | Val loss=1.0986 acc=0.5000
Epoch 017/200 | Train loss=1.0987 acc=0.4997 | Val loss=1.0987 acc=0.5000
Epoch 018/200 | Train loss=1.0986 acc=0.5001 | Val loss=1.0987 acc=0.5000

with 4 blocks:
Epoch 001/200 | Train loss=1.1505 acc=0.3389 | Val loss=1.1059 acc=0.2457
  New best (0.2457) saved to /kaggle/working/models/labram_probe/labram_probe_best.pth
Epoch 002/200 | Train loss=1.1035 acc=0.3532 | Val loss=1.1044 acc=0.5000
  New best (0.5000) saved to /kaggle/working/models/labram_probe/labram_probe_best.pth
Epoch 003/200 | Train loss=1.1018 acc=0.3513 | Val loss=1.0996 acc=0.2543
Epoch 004/200 | Train loss=1.1002 acc=0.2946 | Val loss=1.0989 acc=0.5000
Epoch 005/200 | Train loss=1.0993 acc=0.3279 | Val loss=1.0989 acc=0.2457
Epoch 006/200 | Train loss=1.0993 acc=0.3907 | Val loss=1.0987 acc=0.5000
Epoch 007/200 | Train loss=1.0988 acc=0.3175 | Val loss=1.0987 acc=0.5000
Epoch 008/200 | Train loss=1.0991 acc=0.2967 | Val loss=1.0988 acc=0.2457
Epoch 009/200 | Train loss=1.0988 acc=0.4218 | Val loss=1.0987 acc=0.2457
Epoch 010/200 | Train loss=1.0988 acc=0.3952 | Val loss=1.0987 acc=0.5000
Epoch 011/200 | Train loss=1.0987 acc=0.4996 | Val loss=1.0987 acc=0.5000
\end{verbatim}
\normalsize

\subsection{WeightedRandomSampler Error Traceback}
\scriptsize
\begin{verbatim}
---------------------------------------------------------------------------
IndexError                                Traceback (most recent call last)
/tmp/ipykernel_55/1629047982.py in <cell line: 0>()
    208     val_losses, val_accs = [], []
    209     with torch.no_grad():
--> 210         for X_batch, y_batch in val_loader:
    211             X_batch = X_batch.to(device)
    212             y_batch = y_batch.to(device)

/usr/local/lib/python3.12/dist-packages/torch/utils/data/dataloader.py in __next__(self)
    730                 # TODO(https://github.com/pytorch/pytorch/issues/76750)
    731                 self._reset()  # type: ignore[call-arg]
--> 732             data = self._next_data()
    733             self._num_yielded += 1
    734             if (

/usr/local/lib/python3.12/dist-packages/torch/utils/data/dataloader.py in _next_data(self)
    786     def _next_data(self):
    787         index = self._next_index()  # may raise StopIteration
--> 788         data = self._dataset_fetcher.fetch(index)  # may raise StopIteration
    789         if self._pin_memory:
    790             data = _utils.pin_memory.pin_memory(data, self._pin_memory_device)

/usr/local/lib/python3.12/dist-packages/torch/utils/data/_utils/fetch.py in fetch(self, possibly_batched_index)
     50                 data = self.dataset.__getitems__(possibly_batched_index)
     51             else:
---> 52                 data = [self.dataset[idx] for idx in possibly_batched_index]
     53         else:
     54             data = self.dataset[possibly_batched_index]

/usr/local/lib/python3.12/dist-packages/torch/utils/data/dataset.py in __getitem__(self, index)
    205
    206     def __getitem__(self, index):
--> 207         return tuple(tensor[index] for tensor in self.tensors)
    208
    209     def __len__(self):

/usr/local/lib/python3.12/dist-packages/torch/utils/data/dataset.py in <genexpr>(.0)
    205
    206     def __getitem__(self, index):
--> 207         return tuple(tensor[index] for tensor in self.tensors)
    208
    209     def __len__(self):

IndexError: index 2676 is out of bounds for dimension 0 with size 900
\end{verbatim}
\normalsize

\subsection{Run D: 3-Class Diagnostic Run (Counts + Pred Distributions)}
\scriptsize
\begin{verbatim}
============================================================
TRAINING LaBraM PROBE
============================================================
  Train class counts: [884.0, 871.0, 1755.0]
  Val class counts:   [229.0, 221.0, 450.0]

  Device: cuda
  Trainable params | head=52,227 encoder=966,160 (unfreeze_last_n_blocks=2)
  Optimizer LRs | head=0.001 encoder=1e-05 weight_decay=0.01
  Epochs: 200  |  Early-stop patience: 40

Epoch 001/200 | Train loss=1.1433 acc=0.3379 | Val loss=1.0696 acc=0.5000
  Val true counts=[229, 221, 450] pred counts=[0, 0, 900]
  Val per-class acc=[0.0, 0.0, 1.0]
  New best (0.5000) saved to /kaggle/working/models/labram_probe/labram_probe_best.pth
Epoch 002/200 | Train loss=1.1067 acc=0.3330 | Val loss=1.0779 acc=0.5000
  Val true counts=[229, 221, 450] pred counts=[0, 0, 900]
  Val per-class acc=[0.0, 0.0, 1.0]
Epoch 003/200 | Train loss=1.1010 acc=0.3410 | Val loss=1.0949 acc=0.2544
  Val true counts=[229, 221, 450] pred counts=[900, 0, 0]
  Val per-class acc=[1.0, 0.0, 0.0]
Epoch 004/200 | Train loss=1.0996 acc=0.3376 | Val loss=1.1028 acc=0.2456
  Val true counts=[229, 221, 450] pred counts=[0, 900, 0]
  Val per-class acc=[0.0, 1.0, 0.0]
Epoch 005/200 | Train loss=1.0994 acc=0.3387 | Val loss=1.0998 acc=0.2456
  Val true counts=[229, 221, 450] pred counts=[0, 900, 0]
  Val per-class acc=[0.0, 1.0, 0.0]
Epoch 006/200 | Train loss=1.0990 acc=0.3274 | Val loss=1.0971 acc=0.2456
  Val true counts=[229, 221, 450] pred counts=[0, 900, 0]
  Val per-class acc=[0.0, 1.0, 0.0]
Epoch 007/200 | Train loss=1.0986 acc=0.3422 | Val loss=1.0985 acc=0.2456
  Val true counts=[229, 221, 450] pred counts=[0, 900, 0]
  Val per-class acc=[0.0, 1.0, 0.0]
Epoch 008/200 | Train loss=1.0985 acc=0.3396 | Val loss=1.0982 acc=0.2456
  Val true counts=[229, 221, 450] pred counts=[0, 900, 0]
  Val per-class acc=[0.0, 1.0, 0.0]
Epoch 009/200 | Train loss=1.0988 acc=0.3353 | Val loss=1.0986 acc=0.2456
  Val true counts=[229, 221, 450] pred counts=[0, 900, 0]
  Val per-class acc=[0.0, 1.0, 0.0]
Epoch 010/200 | Train loss=1.0990 acc=0.3299 | Val loss=1.0833 acc=0.5000
  Val true counts=[229, 221, 450] pred counts=[0, 0, 900]
  Val per-class acc=[0.0, 0.0, 1.0]
\end{verbatim}
\normalsize

\subsection{Run E: Binary Mode (T1/T2) with Diagnostic Output}
\scriptsize
\begin{verbatim}
============================================================
TRAINING LaBraM PROBE
============================================================
  Binary mode enabled (T1/T2 only). Kept labels=[0, 1], remap={0: 0, 1: 1}
  Train class counts: [884.0, 871.0]
  Val class counts:   [229.0, 221.0]
  Overrides | pooling=cls unfreeze_last_n_blocks=0 head_lr=0.0001 encoder_lr=1e-05

  Device: cuda

[Sanity] Running frozen-embedding linear probe...
[Sanity] Feature shapes train=(1755, 200) val=(450, 200)
[Sanity] Epoch 01/25 train_loss=1.2102 train_acc=0.5048 val_acc=0.5089
[Sanity] Val true counts=[229, 221] pred counts=[450, 0]
[Sanity] Epoch 05/25 train_loss=0.9566 train_acc=0.5003 val_acc=0.5089
[Sanity] Val true counts=[229, 221] pred counts=[450, 0]
[Sanity] Epoch 10/25 train_loss=0.8089 train_acc=0.5100 val_acc=0.5089
[Sanity] Val true counts=[229, 221] pred counts=[450, 0]
[Sanity] Epoch 15/25 train_loss=0.7810 train_acc=0.5060 val_acc=0.4911
[Sanity] Val true counts=[229, 221] pred counts=[14, 436]
[Sanity] Epoch 20/25 train_loss=0.7993 train_acc=0.5140 val_acc=0.5089
[Sanity] Val true counts=[229, 221] pred counts=[450, 0]
[Sanity] Epoch 25/25 train_loss=0.7463 train_acc=0.5322 val_acc=0.5089
[Sanity] Val true counts=[229, 221] pred counts=[450, 0]
[Sanity] Best val acc: 0.5111

  Trainable params | head=51,970 encoder=0 (unfreeze_last_n_blocks=0)
  Optimizer LR  | head=0.0001 weight_decay=0.01
  Epochs: 200  |  Early-stop patience: 40

Epoch 001/200 | Train loss=0.8636 acc=0.5088 | Val loss=0.7030 acc=0.5089
  Val true counts=[229, 221] pred counts=[450, 0]
  Val per-class acc=[1.0, 0.0]
  New best (0.5089) saved to /kaggle/working/models/labram_probe/labram_probe_best.pth
Epoch 002/200 | Train loss=0.8154 acc=0.5088 | Val loss=0.6962 acc=0.4933
  Val true counts=[229, 221] pred counts=[311, 139]
  Val per-class acc=[0.6812227368354797, 0.2986425459384918]
Epoch 003/200 | Train loss=0.7915 acc=0.5202 | Val loss=0.6965 acc=0.4933
  Val true counts=[229, 221] pred counts=[385, 65]
  Val per-class acc=[0.8427947759628296, 0.1312217265367508]
Epoch 004/200 | Train loss=0.7586 acc=0.5157 | Val loss=0.6965 acc=0.4756
  Val true counts=[229, 221] pred counts=[91, 359]
  Val per-class acc=[0.18340611457824707, 0.7782805562019348]
Epoch 005/200 | Train loss=0.7461 acc=0.5060 | Val loss=0.7042 acc=0.4911
  Val true counts=[229, 221] pred counts=[0, 450]
  Val per-class acc=[0.0, 1.0]
Epoch 006/200 | Train loss=0.7417 acc=0.5014 | Val loss=0.6994 acc=0.4933
  Val true counts=[229, 221] pred counts=[5, 445]
  Val per-class acc=[0.013100436888635159, 0.9909502267837524]
Epoch 007/200 | Train loss=0.7348 acc=0.5054 | Val loss=0.6965 acc=0.4978
  Val true counts=[229, 221] pred counts=[315, 135]
  Val per-class acc=[0.6943231225013733, 0.29411765933036804]
Epoch 008/200 | Train loss=0.7140 acc=0.5265 | Val loss=0.6982 acc=0.5000
  Val true counts=[229, 221] pred counts=[24, 426]
  Val per-class acc=[0.061135370284318924, 0.9547511339187622]
Epoch 009/200 | Train loss=0.7264 acc=0.4963 | Val loss=0.6954 acc=0.5067
  Val true counts=[229, 221] pred counts=[361, 89]
  Val per-class acc=[0.8034934401512146, 0.19909502565860748]
Epoch 010/200 | Train loss=0.7089 acc=0.4997 | Val loss=0.6957 acc=0.4689
  Val true counts=[229, 221] pred counts=[94, 356]
  Val per-class acc=[0.18340611457824707, 0.7647058963775635]
Epoch 011/200 | Train loss=0.7093 acc=0.5083 | Val loss=0.6990 acc=0.5089
  Val true counts=[229, 221] pred counts=[450, 0]
  Val per-class acc=[1.0, 0.0]
Epoch 012/200 | Train loss=0.7064 acc=0.5083 | Val loss=0.6987 acc=0.4911
  Val true counts=[229, 221] pred counts=[0, 450]
  Val per-class acc=[0.0, 1.0]
Epoch 013/200 | Train loss=0.7060 acc=0.4986 | Val loss=0.7060 acc=0.4911
  Val true counts=[229, 221] pred counts=[0, 450]
  Val per-class acc=[0.0, 1.0]
Epoch 014/200 | Train loss=0.7057 acc=0.5009 | Val loss=0.6998 acc=0.4911
  Val true counts=[229, 221] pred counts=[0, 450]
  Val per-class acc=[0.0, 1.0]
Epoch 015/200 | Train loss=0.7013 acc=0.5071 | Val loss=0.6951 acc=0.5089
  Val true counts=[229, 221] pred counts=[188, 262]
  Val per-class acc=[0.42794761061668396, 0.5927602052688599]
Epoch 016/200 | Train loss=0.7031 acc=0.5123 | Val loss=0.6990 acc=0.4911
  Val true counts=[229, 221] pred counts=[0, 450]
  Val per-class acc=[0.0, 1.0]
Epoch 017/200 | Train loss=0.6975 acc=0.5202 | Val loss=0.6963 acc=0.5067
  Val true counts=[229, 221] pred counts=[449, 1]
  Val per-class acc=[0.9956331849098206, 0.0]
Epoch 018/200 | Train loss=0.6975 acc=0.4883 | Val loss=0.6982 acc=0.4911
  Val true counts=[229, 221] pred counts=[0, 450]
  Val per-class acc=[0.0, 1.0]
Epoch 019/200 | Train loss=0.6976 acc=0.4969 | Val loss=0.6950 acc=0.5067
  Val true counts=[229, 221] pred counts=[449, 1]
  Val per-class acc=[0.9956331849098206, 0.0]
Epoch 020/200 | Train loss=0.7004 acc=0.5094 | Val loss=0.7073 acc=0.5089
  Val true counts=[229, 221] pred counts=[450, 0]
  Val per-class acc=[1.0, 0.0]
Epoch 021/200 | Train loss=0.6989 acc=0.5037 | Val loss=0.6961 acc=0.5089
  Val true counts=[229, 221] pred counts=[450, 0]
  Val per-class acc=[1.0, 0.0]
Epoch 022/200 | Train loss=0.6948 acc=0.5191 | Val loss=0.6967 acc=0.4911
  Val true counts=[229, 221] pred counts=[2, 448]
  Val per-class acc=[0.0043668122962117195, 0.9954751133918762]
Epoch 023/200 | Train loss=0.6953 acc=0.5151 | Val loss=0.6982 acc=0.5089
  Val true counts=[229, 221] pred counts=[450, 0]
  Val per-class acc=[1.0, 0.0]
Epoch 024/200 | Train loss=0.6982 acc=0.4963 | Val loss=0.6941 acc=0.5089
  Val true counts=[229, 221] pred counts=[450, 0]
  Val per-class acc=[1.0, 0.0]
Epoch 025/200 | Train loss=0.6997 acc=0.4997 | Val loss=0.6954 acc=0.5089
  Val true counts=[229, 221] pred counts=[450, 0]
  Val per-class acc=[1.0, 0.0]
Epoch 026/200 | Train loss=0.6961 acc=0.5014 | Val loss=0.7014 acc=0.4911
  Val true counts=[229, 221] pred counts=[0, 450]
  Val per-class acc=[0.0, 1.0]
Epoch 027/200 | Train loss=0.6928 acc=0.5214 | Val loss=0.6972 acc=0.5089
  Val true counts=[229, 221] pred counts=[450, 0]
  Val per-class acc=[1.0, 0.0]
Epoch 028/200 | Train loss=0.6946 acc=0.5105 | Val loss=0.6985 acc=0.4911
  Val true counts=[229, 221] pred counts=[0, 450]
  Val per-class acc=[0.0, 1.0]
Epoch 029/200 | Train loss=0.6946 acc=0.4912 | Val loss=0.6952 acc=0.4844
  Val true counts=[229, 221] pred counts=[15, 435]
  Val per-class acc=[0.026200873777270317, 0.959276020526886]
Epoch 030/200 | Train loss=0.6938 acc=0.5174 | Val loss=0.6987 acc=0.4911
  Val true counts=[229, 221] pred counts=[0, 450]
  Val per-class acc=[0.0, 1.0]
Epoch 031/200 | Train loss=0.6989 acc=0.4849 | Val loss=0.6966 acc=0.4911
  Val true counts=[229, 221] pred counts=[0, 450]
  Val per-class acc=[0.0, 1.0]
Epoch 032/200 | Train loss=0.6949 acc=0.5123 | Val loss=0.6941 acc=0.4844
  Val true counts=[229, 221] pred counts=[25, 425]
  Val per-class acc=[0.04803493618965149, 0.9366515874862671]
Epoch 033/200 | Train loss=0.6922 acc=0.5271 | Val loss=0.6952 acc=0.5089
  Val true counts=[229, 221] pred counts=[450, 0]
  Val per-class acc=[1.0, 0.0]
Epoch 034/200 | Train loss=0.6969 acc=0.4838 | Val loss=0.6940 acc=0.5111
  Val true counts=[229, 221] pred counts=[97, 353]
  Val per-class acc=[0.23144105076789856, 0.8009049892425537]
  New best (0.5111) saved to /kaggle/working/models/labram_probe/labram_probe_best.pth
Epoch 035/200 | Train loss=0.6932 acc=0.5031 | Val loss=0.6940 acc=0.5111
  Val true counts=[229, 221] pred counts=[449, 1]
  Val per-class acc=[1.0, 0.004524887073785067]
Epoch 036/200 | Train loss=0.6929 acc=0.5259 | Val loss=0.6951 acc=0.4933
  Val true counts=[229, 221] pred counts=[1, 449]
  Val per-class acc=[0.0043668122962117195, 1.0]
Epoch 037/200 | Train loss=0.6919 acc=0.5379 | Val loss=0.6936 acc=0.5044
  Val true counts=[229, 221] pred counts=[444, 6]
  Val per-class acc=[0.9825327396392822, 0.009049774147570133]
Epoch 038/200 | Train loss=0.6939 acc=0.5100 | Val loss=0.6944 acc=0.4867
  Val true counts=[229, 221] pred counts=[40, 410]
  Val per-class acc=[0.08296943455934525, 0.9049773812294006]
Epoch 039/200 | Train loss=0.6950 acc=0.5123 | Val loss=0.6943 acc=0.5178
  Val true counts=[229, 221] pred counts=[126, 324]
  Val per-class acc=[0.3013100326061249, 0.7420814633369446]
  New best (0.5178) saved to /kaggle/working/models/labram_probe/labram_probe_best.pth
Epoch 040/200 | Train loss=0.6947 acc=0.5020 | Val loss=0.6954 acc=0.4889
  Val true counts=[229, 221] pred counts=[1, 449]
  Val per-class acc=[0.0, 0.9954751133918762]
Epoch 041/200 | Train loss=0.6935 acc=0.5151 | Val loss=0.6934 acc=0.5089
  Val true counts=[229, 221] pred counts=[444, 6]
  Val per-class acc=[0.9868995547294617, 0.013574660755693913]
Epoch 042/200 | Train loss=0.6931 acc=0.5117 | Val loss=0.6943 acc=0.4911
  Val true counts=[229, 221] pred counts=[2, 448]
  Val per-class acc=[0.0043668122962117195, 0.9954751133918762]
\end{verbatim}
\normalsize

\subsection{Run F: Binary Mode After Removing Weighted Sampler}
\scriptsize
\begin{verbatim}
============================================================
TRAINING LaBraM PROBE
============================================================
  Binary mode enabled (T1/T2 only). Kept labels=[0, 1], remap={0: 0, 1: 1}
  Train class counts: [884.0, 871.0]
  Val class counts:   [229.0, 221.0]
  Overrides | pooling=cls unfreeze_last_n_blocks=0 head_lr=0.0001 encoder_lr=1e-05

  Device: cuda

[Sanity] Running frozen-embedding linear probe...
[Sanity] Feature shapes train=(1755, 200) val=(450, 200)
[Sanity] Epoch 01/25 train_loss=0.9123 train_acc=0.4860 val_acc=0.5089
[Sanity] Val true counts=[229, 221] pred counts=[74, 376]
[Sanity] Epoch 05/25 train_loss=0.7581 train_acc=0.5185 val_acc=0.4889
[Sanity] Val true counts=[229, 221] pred counts=[1, 449]
[Sanity] Epoch 10/25 train_loss=0.7403 train_acc=0.5037 val_acc=0.5133
[Sanity] Val true counts=[229, 221] pred counts=[172, 278]
[Sanity] Epoch 15/25 train_loss=0.7156 train_acc=0.5282 val_acc=0.5089
[Sanity] Val true counts=[229, 221] pred counts=[450, 0]
[Sanity] Epoch 20/25 train_loss=0.7585 train_acc=0.5276 val_acc=0.4978
[Sanity] Val true counts=[229, 221] pred counts=[419, 31]
[Sanity] Epoch 25/25 train_loss=0.8477 train_acc=0.5481 val_acc=0.4911
[Sanity] Val true counts=[229, 221] pred counts=[0, 450]
[Sanity] Best val acc: 0.5267

  Trainable params | head=51,970 encoder=0 (unfreeze_last_n_blocks=0)
  Optimizer LR  | head=0.0001 weight_decay=0.01
  Epochs: 200  |  Early-stop patience: 40

Epoch 001/200 | Train loss=0.8512 acc=0.4929 | Val loss=0.6916 acc=0.5289
  Val true counts=[229, 221] pred counts=[131, 319]
  Val per-class acc=[0.3231441080570221, 0.7420814633369446]
  New best (0.5289) saved to /kaggle/working/models/labram_probe/labram_probe_best.pth
Epoch 002/200 | Train loss=0.7797 acc=0.5060 | Val loss=0.7054 acc=0.4911
  Val true counts=[229, 221] pred counts=[0, 450]
  Val per-class acc=[0.0, 1.0]
Epoch 003/200 | Train loss=0.7725 acc=0.4997 | Val loss=0.7047 acc=0.4911
  Val true counts=[229, 221] pred counts=[0, 450]
  Val per-class acc=[0.0, 1.0]
Epoch 004/200 | Train loss=0.7447 acc=0.5145 | Val loss=0.6918 acc=0.5378
  Val true counts=[229, 221] pred counts=[183, 267]
  Val per-class acc=[0.44541484117507935, 0.6334841847419739]
  New best (0.5378) saved to /kaggle/working/models/labram_probe/labram_probe_best.pth
Epoch 005/200 | Train loss=0.7568 acc=0.5026 | Val loss=0.7316 acc=0.5089
  Val true counts=[229, 221] pred counts=[450, 0]
  Val per-class acc=[1.0, 0.0]
Epoch 006/200 | Train loss=0.7458 acc=0.4963 | Val loss=0.6944 acc=0.5089
  Val true counts=[229, 221] pred counts=[450, 0]
  Val per-class acc=[1.0, 0.0]
Epoch 007/200 | Train loss=0.7251 acc=0.4969 | Val loss=0.7080 acc=0.5089
  Val true counts=[229, 221] pred counts=[450, 0]
  Val per-class acc=[1.0, 0.0]
Epoch 008/200 | Train loss=0.7123 acc=0.5123 | Val loss=0.6939 acc=0.4889
  Val true counts=[229, 221] pred counts=[9, 441]
  Val per-class acc=[0.017467249184846878, 0.9773755669593811]
Epoch 009/200 | Train loss=0.7184 acc=0.4980 | Val loss=0.6941 acc=0.4956
  Val true counts=[229, 221] pred counts=[14, 436]
  Val per-class acc=[0.034934498369693756, 0.9728506803512573]
Epoch 010/200 | Train loss=0.7059 acc=0.5100 | Val loss=0.6982 acc=0.5089
  Val true counts=[229, 221] pred counts=[450, 0]
  Val per-class acc=[1.0, 0.0]
Epoch 011/200 | Train loss=0.7098 acc=0.5088 | Val loss=0.6935 acc=0.5111
  Val true counts=[229, 221] pred counts=[115, 335]
  Val per-class acc=[0.27074235677719116, 0.7601810097694397]
Epoch 012/200 | Train loss=0.6998 acc=0.4974 | Val loss=0.6929 acc=0.4822
  Val true counts=[229, 221] pred counts=[356, 94]
  Val per-class acc=[0.7685589790344238, 0.18552036583423615]
Epoch 013/200 | Train loss=0.7011 acc=0.5128 | Val loss=0.6940 acc=0.5089
  Val true counts=[229, 221] pred counts=[450, 0]
  Val per-class acc=[1.0, 0.0]
Epoch 014/200 | Train loss=0.7003 acc=0.5048 | Val loss=0.6929 acc=0.5156
  Val true counts=[229, 221] pred counts=[409, 41]
  Val per-class acc=[0.9170305728912354, 0.09954751282930374]
Epoch 015/200 | Train loss=0.7064 acc=0.4769 | Val loss=0.6941 acc=0.5089
  Val true counts=[229, 221] pred counts=[450, 0]
  Val per-class acc=[1.0, 0.0]
Epoch 016/200 | Train loss=0.7053 acc=0.4940 | Val loss=0.6997 acc=0.5089
  Val true counts=[229, 221] pred counts=[450, 0]
  Val per-class acc=[1.0, 0.0]
Epoch 017/200 | Train loss=0.6953 acc=0.5111 | Val loss=0.6934 acc=0.4733
  Val true counts=[229, 221] pred counts=[82, 368]
  Val per-class acc=[0.16157205402851105, 0.7963801026344299]
Epoch 018/200 | Train loss=0.7021 acc=0.4815 | Val loss=0.6934 acc=0.5133
  Val true counts=[229, 221] pred counts=[430, 20]
  Val per-class acc=[0.960698664188385, 0.04977375641465187]
\end{verbatim}
\normalsize




\end{document}
